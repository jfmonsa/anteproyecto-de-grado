\section{Impacto Ambiental}

Dada la importancia de evaluar el impacto ambiental de las actividades del proyecto. Como se puede observar en la \autoref{tab:consumo_energetico_dispositivos} se procede a realizar un análisis de la huella de carbono generada por el consumo energético de los dispositivos utilizados en la operación.

% Avoid magic strings
\newcommand{\deviceAName}{ Computador principal }
\newcommand{\deviceBName}{ Monitor Sansui ES-22X3 }
\newcommand{\deviceCName}{ Conexión a internet }

\begin{table}[H]
  \centering
  \begin{tabular}{|c|c|c|c|}
    \hline
    \grayTableHeaderCell{3cm}{Dispositivo} &
    \grayTableHeaderCell{3cm}{Consumo Real Promedio (Watts / hora)} &
    \grayTableHeaderCell{3cm}{Consumo en Suspensión Promedio (Watts / hora)} &
    \grayTableHeaderCell{3cm}{Consumo Apagado Promedio (Watts / hora)} \\
    \hline

    \deviceAName\footnote{El consumo del computador principal fue medido usando la herramienta \texttt{powerstat} en Linux (polyfiling).}  & 11,63 & 2,5 & 5 \\

    \deviceBName & 13,35 \cite{Device_report_sansui} & 0,5 & 1 \\
    \deviceCName & 25 & 0 & 0 \\
    \hline

    Total & 49,98 & 3,0 & 6,0 \\
    \hline
  \end{tabular}
  \caption{Consumo energético de los dispositivos. Fuente: Elaboración propia.}
  \label{tab:consumo_energetico_dispositivos}
\end{table}

Por otra parte, la \autoref{tab:consumo_energetico_previsto} presenta el consumo de cada dispositivo con base en las horas efectivas de trabajo previstas para las fases iniciales de la operación.

\begin{table}[H]
  \centering
  \begin{tabular}{|p{3cm}|c|c|c|c|c|c|c|c|}
    \hline
    \grayTableHeaderCell{2cm}{Dispositivo} &
    \grayTableHeaderCell{2cm}{Uso normal horas / día} &
    \grayTableHeaderCell{1cm}{ (1) Kw / hora } &
    \grayTableHeaderCell{1cm}{ (2) Horas / día }  &
    \grayTableHeaderCell{1cm}{ (3) Kw / hora }  &
    \grayTableHeaderCell{1cm}{ (4) Horas / día } &
    \grayTableHeaderCell{1cm}{ (5) Kw / hora } &
    \grayTableHeaderCell{1cm}{ (6) Kw / hora } &
    \grayTableHeaderCell{1cm}{ (7) Kw / hora } \\
    \hline

    \deviceAName & 6 & 0.100 & 1 & 0.005 & 17 & 0.001 & 0.600 & 0.606 \\
    \deviceBName & 6 & 0.025 & 1 & 0.002 & 17 & 0.001 & 0.150 & 0.153 \\
    \deviceCName & 24 & 0.010 & 0 & 0.000 & 0 & 0.001 & 0.240 & 0.240 \\

    \hline
    \textbf{Total día} & - & - & - & - & - & - & \textbf{0.990} & \textbf{0.999} \\
    \hline
  \end{tabular}
  \caption{Consumo energético previsto de los dispositivos. Fuente: Elaboración propia.}
  \label{tab:consumo_energetico_previsto}
\end{table}

\begin{itemize}
  \item (1) Consumo en condiciones de funcionamiento normal
  \item (2) Consumo en estado de suspensión o hibernado
  \item (3) Consumo en estado de apagado
  \item (4) Consumo real total al día
  \item (5) Consumo fantasma al día
\end{itemize}

Suponiendo que por semana se trabajan 5 días, se presentan los consumos semanales, fantasma, mensual y anual en la \autoref{tab:consumo_energetico_totalizado}.

\begin{table}[H]
  \centering
  \begin{tabular}{|c|c|c|c|c|}
    \hline
    \grayTableHeaderCell{3cm}{Consumo Semanal (Kw/hora)} &
    \grayTableHeaderCell{3cm}{Consumo Fantasma (Kw/hora)} &
    \grayTableHeaderCell{3cm}{Consumo Mensual (Kw/hora)} &
    \grayTableHeaderCell{3cm}{Consumo Anual (Kw/hora)} \\
    \hline
    6,993 & 0,27 & 29,97 & 364,64 \\
    \hline
  \end{tabular}
  \caption{Consumo energético totalizado. Fuente: Elaboración propia.}
  \label{tab:consumo_energetico_totalizado}
\end{table}

Por otro lado, para calcular la huella de carbono generada por el consumo energético de los dispositivos, se emplea la siguiente ecuación \autoref{eq:huella_de_carbono}, basada en la metodología propuesta por \textcite{kean2012}  y la \textcite{upme2019factor}:

\begin{align}
  \label{eq:huella_de_carbono}
  \text{Huella de Carbono (Tn CO$_2$e)} =\ & \text{Consumo Anual (GWh)} \notag \\
  & \times\ \text{Factor de Emisión (Tn CO$_2$e / GWh)} \notag \\
  & \times\ \text{Potencial de Calentamiento Global}
\end{align}

Esta fórmula permite estimar las emisiones de gases de efecto invernadero expresadas en toneladas de $CO_2$ equivalente ($Tn$ $CO_2e$), a partir del consumo eléctrico anual, el factor de emisión del sistema interconectado nacional (SIN) y el potencial de calentamiento global del gas considerado.

\begin{table}[H]
  \centering
  \begin{tabular}{|c|c|c|c|}
    \hline
    \grayTableHeaderCell{3cm}{Factor de Emisión ($Tn$ $CO_2e$ / $Gwh$)} &
    \grayTableHeaderCell{3cm}{Consumo Anual ($Gwh$)} &
    \grayTableHeaderCell{3cm}{Potencial de Calentamiento Global} &
    \grayTableHeaderCell{3cm}{Huella de Carbono ($Tn$ $CO_2e$)} \\
    \hline
    0,13 & 2,95884 & 1 & 0,3846 \\
    \hline
  \end{tabular}
  \caption{Huella de carbono generada por el consumo energético. Fuente: Elaboración propia.}
  \label{tab:huella_de_carbono}
\end{table}

Con base en la \autoref{tab:huella_de_carbono}, se estima que la cantidad de dióxido de carbono equivalente ($CO_2e$) generada por el consumo energético anual es de aproximadamente 0,38465 toneladas. Dado que este valor resulta ser bajo en comparación con los umbrales comúnmente establecidos en la literatura para justificar procesos de compensación, se opta por implementar acciones orientadas a la reducción del consumo energético. Algunas medidas recomendadas son las siguientes:

\begin{itemize}
  \item Apagar el computador y el monitor en lugar de dejarlos en estado de hibernación.
  \item Desconectar el computador y el monitor de la fuente de energía al finalizar la jornada laboral.
  \item Desconectar el teléfono fijo cuando no se encuentre en uso o al retirarse de la oficina.
  \item Desconectar el router al finalizar la jornada o al dejar la oficina por un periodo prolongado.
\end{itemize}
