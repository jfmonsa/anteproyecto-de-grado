\section{Introducción}

En Colombia, el acceso a la información judicial se ha centralizado mediante el Portal  “Rama Judicial Unificada”, una página web donde cualquier ciudadano puede consultar el estado de los procesos legales. Sin embargo, este sistema no cuenta con mecanismos automatizados de seguimiento o notificación de cambios, lo cual obliga a los profesionales del derecho a revisar constantemente el estado de cada proceso de forma manual \cite{Paez2024}. Sumado a esto,  según el Consejo Superior de la Judicatura para el 2023 el índice de congestión de la justicia en Colombia fue de 54.9\% \cite{Rozo2024}. Esta situación, especialmente crítica para abogados que gestionan múltiples casos, no solo representa una carga operativa considerable, sino que también incrementa el riesgo de omitir actualizaciones importantes que pueden afectar el desarrollo de los casos.

Ante este panorama, se abre una clara oportunidad de negocio para desarrollar software que automatice y optimice esta tarea, además de ofrecer funcionalidades adicionales que permitan aprovechar este nicho de mercado. Todo esto, de la mano de tecnologías emergentes como la inteligencia artificial, la computación, entre otras. Estas tecnologías permiten no solo reducir la carga operativa del profesional jurídico, sino también mejorar la calidad del servicio prestado al cliente \cite{Botero2023}. En este contexto, el sector \textit{Legaltech} ha ganado tracción en Colombia y Latinoamérica, Siendo Colombia quien ocupa el segundo lugar en cuanto a productos de software en la industria jurídica según el Legal Tech Index \cite{Sierra2023}.

En este marco surge Paralegales, una plataforma desarrollada por la empresa Webcloster S.A.S., cuyo propósito es automatizar el seguimiento de procesos judiciales y brindar a los abogados una experiencia más eficiente, inteligente y centralizada. Este proyecto de grado tiene como objetivo migrar la infraestructura actual de Paralegales, basada parcialmente en Firebase, hacia un entorno 100\% en AWS. Esta migración permitirá mejorar la seguridad de la lógica de negocio, actualmente expuesta en el cliente (\textit{frontend}), aumentar la escalabilidad del sistema y reducir los costos operativos. Además, se buscará aplicar las mejores prácticas de ingeniería de software dentro de lo posible para fortalecer la calidad técnica del producto durante su evolución.
