\subsection{Marco Teórico}

\subsubsection{Teoría General de Sistemas}

La Teoría General de Sistemas (TGS), propuesta por Ludwig von Bertalanffy en la década de 1950, plantea que cualquier organización, fenómeno o entidad puede ser comprendida como un sistema compuesto por partes interrelacionadas que funcionan como un todo para lograr una serie de objetivos. Desde esta perspectiva, un sistema no se analiza en forma aislada, sino en términos de sus relaciones internas y su interacción con el entorno \cite{Bertalanffy1968}.

Según la TGS, todo sistema está compuesto por subsistemas interdependientes que interactúan entre sí, y a su vez forma parte de un suprasistema o entorno más amplio. Estos subsistemas intercambian información, recursos o energía a través de entradas (\textit{inputs}) y salidas (\textit{outputs}), y buscan mantener un estado de equilibrio dinámico para adaptarse a cambios en el entorno \cite{Bertalanffy1968}. Los principios de retroalimentación, homeostasis, equifinalidad y sinergia son claves para entender el comportamiento sistémico.

En este contexto, Paralegales como proyecto de Software puede concebirse como un sistema socio-técnico complejo, compuesto por diversos subsistemas funcionales: el \textit{backend}, el \textit{frontend}, la base de datos, la infraestructura en la nube, los usuarios y el equipo de desarrolladores, los stakeholders, etc.

La decisión de migrar de Firebase a AWS se fundamenta, desde la TGS, en la necesidad de reconfigurar el sistema para mejorar su capacidad de adaptación y evolución. Al adoptar modelos como BaaS, FaaS e infraestructura modular que reemplacen la dependencia de Firebase, se busca lograr una arquitectura más abierta, flexible y mantenible, que facilite la interacción eficiente entre los subsistemas y responda mejor a los cambios del entorno organizacional y tecnológico.

Asimismo, el enfoque sistémico permite identificar interdependencias críticas, como la coordinación entre equipos, la automatización del despliegue de software (CI/CD), la integración de servicios distribuidos. En la organización del código fuente, permite analizar las dependencias entre los componentes, abstracciones, acoplamiento, etc. Todo esto permite al sistema Paralegales alcanzar una mayor resiliencia y avanzar hacia un estado de equilibrio más robusto y sostenible.
