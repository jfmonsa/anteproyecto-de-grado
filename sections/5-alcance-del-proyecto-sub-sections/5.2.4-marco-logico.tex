\subsubsection{Marco Lógico de la Propuesta}

% --- Objective A ---
\newcommand{\objetivoEspecificoAActividades}{
  \begin{itemize}[leftmargin=*, topsep=0pt, itemsep=0pt]
    \item Identificación de lógica crítica actual en el frontend
    \item Diseño de funciones Lambda seguras
    \item Implementación y pruebas de funciones
    \item Despliegue en entorno AWS
  \end{itemize}
}

\newcommand{\objetivoEspecificoAMetodologia}{
  Desarrollo ágil basado en SCRUM y prácticas DevOps, con énfasis en integración continua y validaciones automatizadas.
}

\newcommand{\objetivoEspecificoAEntregables}{
  \begin{itemize}[leftmargin=*, topsep=0pt, itemsep=0pt]
    \item Código fuente migrado y desplegado en AWS Lambda
  \end{itemize}
}

\newcommand{\objetivoEspecificoAIndicador}{
  Porcentaje de funciones críticas migradas y en funcionamiento en AWS Lambda
}

\newcommand{\objetivoEspecificoAMeta}{
  100\% de la lógica crítica migrada de frontend a Lambda con cobertura de pruebas y validación de funcionamiento
}

% --- Objective B ---
\newcommand{\objetivoEspecificoBActividades}{
  \begin{itemize}[leftmargin=*, topsep=0pt, itemsep=0pt]
    \item Identificación de dependencias actuales a Firebase
    \item Selección de servicios AWS equivalentes
    \item Refactorización e integración
    \item Validación de funcionamiento
  \end{itemize}
}

\newcommand{\objetivoEspecificoBMetodologia}{
  Iteraciones SCRUM con entregas incrementales y validación de servicios AWS equivalentes.
}

\newcommand{\objetivoEspecificoBEntregables}{
  \begin{itemize}[leftmargin=*, topsep=0pt, itemsep=0pt]
    \item Módulos refactorizados sin dependencia de Firebase
  \end{itemize}
}

\newcommand{\objetivoEspecificoBIndicador}{
  Número de módulos refactorizados y porcentaje de código sin Firebase
}

\newcommand{\objetivoEspecificoBMeta}{
  100\% de los módulos críticos operando sin Firebase y utilizando servicios de AWS
}
% --- Objective C ---
\newcommand{\objetivoEspecificoCActividades}{
  \begin{itemize}[leftmargin=*, topsep=0pt, itemsep=0pt]
    \item Levantamiento de arquitectura actual
    \item Diseño de nueva arquitectura AWS
    \item Implementación de migración de datos y usuarios
    \item Validación y pruebas
  \end{itemize}
}

\newcommand{\objetivoEspecificoCMetodologia}{
  Diseño arquitectónico de infraestructura basado en buenas prácticas de AWS Well-Architected Framework y metodología ETL para migración de datos.
}

\newcommand{\objetivoEspecificoCEntregables}{
  \begin{itemize}[leftmargin=*, topsep=0pt, itemsep=0pt]
    \item Diagrama de arquitectura la pipeline de migración de datos y usuarios hacia AWS.
    \item Documentación de la arquitectura
  \end{itemize}
}

\newcommand{\objetivoEspecificoCIndicador}{
  \begin{itemize}[leftmargin=*, topsep=0pt, itemsep=0pt]
    \item Base de datos y usuarios migrados correctamente
    \item Funcionami- ento de arquitectura desplegada
  \end{itemize}
}

\newcommand{\objetivoEspecificoCMeta}{
  100\% de datos y registros de usuarios migrados y funcionando según pruebas
}

% --- Objective D ---
\newcommand{\objetivoEspecificoDActividades}{
  \begin{itemize}[leftmargin=*, topsep=0pt, itemsep=0pt]
    \item Identificación de componentes críticos a testear
    \item Desarrollo de pruebas automatizadas- Integración con CI/CD
    \item Validación continua
  \end{itemize}
}

\newcommand{\objetivoEspecificoDMetodologia}{
  \textit{Test Driven Development} (TDD) como enfoque principal y cultura DevOps para CI/CD
}

\newcommand{\objetivoEspecificoDEntregables}{
  \begin{itemize}[leftmargin=*, topsep=0pt, itemsep=0pt]
    \item Conjunto de pruebas automatizadas
    \item Pipeline de CI implementada y artefactos sobre la cobertura de código generados en la misma
  \end{itemize}
}

\newcommand{\objetivoEspecificoDIndicador}{
  \begin{itemize}[leftmargin=*, topsep=0pt, itemsep=0pt]
    \item Número de pruebas implementadas y porcentaje de cobertura de pruebas
    \item Pipeline operativa
  \end{itemize}
}

\newcommand{\objetivoEspecificoDMeta}{
  \begin{itemize}[leftmargin=*, topsep=0pt, itemsep=0pt]
    \item 70\% o más de cobertura de pruebas en código refactorizado
    \item Pipeline funcional con ejecución automática
  \end{itemize}
}

% -- Objective Document --
\newcommand{\objetivoEspecificoDocument}{
  Elaborar el documento final del trabajo de grado.
}

\newcommand{\objetivoEspecificoDocumentActividades}{
  \begin{itemize}[leftmargin=*, topsep=0pt, itemsep=0pt]
    \item Redacción progresiva.
    \item Revisión con tutor.
    \item Preparación para entrega y defensa.
  \end{itemize}
}

\newcommand{\objetivoEspecificoDocumentMetodologia}{
  Escritura académica con revisión continua.
}

\newcommand{\objetivoEspecificoDocumentEntregables}{
  Documento completo, corregido ypresentado.
}

\newcommand{\objetivoEspecificoDocumentIndicador}{
  Porcentaje de Capítulos desarrollados y validados por el tutor.
}

\newcommand{\objetivoEspecificoDocumentMeta}{
  100\% de los capítulos aprobados y listos para sustentación
}

% -- Table Setup --
\newcommand{\marcoLogicoHeader}{
  \grayTableHeaderCell{2.5cm}{Objetivo específico} &
  \grayTableHeaderCell{2.5cm}{Actividades} &
  \grayTableHeaderCell{2cm}{Metodología} &
  \grayTableHeaderCell{2.5cm}{Entregables} &
  \grayTableHeaderCell{2cm}{Indicador} &
  \grayTableHeaderCell{2cm}{Meta} \\
}

\renewcommand{\arraystretch}{1.4}
\begingroup
\scriptsize
\begin{longtable}{|
    >{\raggedright\arraybackslash}p{2.5cm}|
    >{\raggedright\arraybackslash}p{2.5cm}|
    >{\raggedright\arraybackslash}p{2cm}|
    >{\raggedright\arraybackslash}p{2.5cm}|
    >{\raggedright\arraybackslash}p{2cm}|
    >{\raggedright\arraybackslash}p{2cm}|
  }
  \hline

  % Table Header
  \marcoLogicoHeader
  \hline
  \endfirsthead

  % Encabezado para las siguientes páginas:
  \hline
  \marcoLogicoHeader
  \hline
  \endhead

  \continuacionTablaFooter{6}
  \endfoot
  \endlastfoot

  % Table Body
  \tableCell\objetivoEspecificoA &
  \tableCell\objetivoEspecificoAActividades &
  \tableCell\objetivoEspecificoAMetodologia &
  \tableCell\objetivoEspecificoAEntregables &
  \tableCell\objetivoEspecificoAIndicador &
  \tableCell\objetivoEspecificoAMeta \\
  \hline

  \tableCell\objetivoEspecificoB &
  \tableCell\objetivoEspecificoBActividades &
  \tableCell\objetivoEspecificoBMetodologia &
  \tableCell\objetivoEspecificoBEntregables &
  \tableCell\objetivoEspecificoBIndicador &
  \tableCell\objetivoEspecificoBMeta \\
  \hline

  %
  \tableCell\objetivoEspecificoC &
  \tableCell\objetivoEspecificoCActividades &
  \tableCell\objetivoEspecificoCMetodologia &
  \tableCell\objetivoEspecificoCEntregables &
  \tableCell\objetivoEspecificoCIndicador &
  \tableCell\objetivoEspecificoCMeta \\
  \hline

  \tableCell\objetivoEspecificoD &
  \tableCell\objetivoEspecificoDActividades &
  \tableCell\objetivoEspecificoDMetodologia &
  \tableCell\objetivoEspecificoDEntregables &
  \tableCell\objetivoEspecificoDIndicador &
  \tableCell\objetivoEspecificoDMeta \\
  \hline

  \tableCell\objetivoEspecificoDocument &
  \tableCell\objetivoEspecificoDocumentActividades &
  \tableCell\objetivoEspecificoDocumentMetodologia &
  \tableCell\objetivoEspecificoDocumentEntregables &
  \tableCell\objetivoEspecificoDocumentIndicador &
  \tableCell\objetivoEspecificoDocumentMeta \\
  \hline

  \caption{Marco Lógico de la Propuesta. Fuente: Elaboración propia.}
  \label{tab:marco_logico}
\end{longtable}
\endgroup
