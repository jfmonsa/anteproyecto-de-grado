\subsubsection{Relación objetivos específicos y resultados esperados}

% avoid nested structures in table
\newcommand\objetivoEspecificoAResulados{
  \begin{itemize}[leftmargin=*, topsep=0pt, itemsep=0pt]
    \item Reducción del riesgo de exposición de lógica sensible al moverla fuera del cliente.
    \item Trazabilidad de la ejecución de la lógica crítica mediante CloudWatch y otros servicios de monitoreo de AWS.
  \end{itemize}
}

\newcommand\objetivoEspecificoBResultados{
  \begin{itemize}[leftmargin=*, topsep=0pt, itemsep=0pt]
    \item Eliminación gradual de acoplamientos con servicios propietarios de Google.
    \item Minimizar el impacto de la migración garantizando una transición controlada.
  \end{itemize}
}

\newcommand\objetivoEspecificoCResultados{
  \begin{itemize}[leftmargin=*, topsep=0pt, itemsep=0pt]
    \item Infraestructura organizada y mantenible sobre AWS, que garantice la integridad y trazabilidad de los datos durante la migración.
    \item Migración exitosa de datos desde Firestore a DynamoDB mediante una pipeline ETL.
    \item Registro de usuarios migrados correctamente a Amazon Cognito con integraciones funcionales.
  \end{itemize}
}

\newcommand\objetivoEspecificoDResultados{
  \begin{itemize}[leftmargin=1em, topsep=2pt, itemsep=2pt]
    \item Las partes del sistema afectadas por la migración cuentan con una cobertura mínima del 70\% en pruebas automatizadas.
    \item Se implementa una pipeline de integración continua (CI) que ejecuta automáticamente las pruebas en cada cambio o despliegue.
    \item Se detectan y corrigen errores de manera temprana gracias a la automatización del proceso de validación del código.
  \end{itemize}
}

% \centering
% \renewcommand{\arraystretch}{1.2}
\begin{longtable}{|p{8cm}|p{8cm}|}
  \hline

  % Table Header
  \grayTableHeaderCell{8cm}{Objetivo específico} &
  \grayTableHeaderCell{8cm}{Resultados esperados} \\
  \hline

  % Table Body
  \tableCell{\objetivoEspecificoA}   &
  \tableCell{\objetivoEspecificoAResulados} \\
  \hline

  \tableCell{\objetivoEspecificoB}   &
  \tableCell{\objetivoEspecificoBResultados} \\
  \hline
  \endfirsthead

  \tableCell{\objetivoEspecificoB}   &
  \tableCell{\objetivoEspecificoBResultados} \\
  \hline
  \endhead

  \continuacionTablaFooter{2}
  \endfoot
  \endlastfoot

  \tableCell{\objetivoEspecificoC}   &
  \tableCell{\objetivoEspecificoCResultados} \\
  \hline

  \tableCell{\objetivoEspecificoD}   &
  \tableCell{\objetivoEspecificoDResultados} \\
  \hline
  \caption{Objetivos específicos y resultados esperados. Fuente: Elaboración propia.}
  \label{tab:objetivos_resultados}
\end{longtable}
