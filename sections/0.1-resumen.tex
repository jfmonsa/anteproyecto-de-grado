\newpage
\section*{Resumen}

Paralegales es una plataforma de legaltech orientada a automatizar el seguimiento de procesos judiciales en la Rama Judicial Unificada en Colombia. El objetivo principal de esta solución es simplificar la tarea de los abogados al ofrecer herramientas que permiten monitorear automáticamente los cambios en los procesos judiciales, además de proporcionar funcionalidades complementarias para mejorar la eficiencia en la gestión legal.

Para acelerar la salida a producción, el desarrollo de Paralegales se hizo uso del Software Development Kit (SDK) de Firebase, lo que permitió avanzar rápidamente en funcionalidades clave. Sin embargo, esta decisión trajo consigo importantes compromisos técnicos, como la implementación de lógica de negocio crítica directamente en el frontend, lo que representa riesgos de seguridad y dificultades para aplicar principios de escalabilidad.

El presente proyecto tiene como objetivo principal migrar y adaptar la infraestructura tecnológica de Paralegales desde Firebase hacia un entorno 100\% en AWS (Amazon Web Services). Esta migración busca mejorar la seguridad del sistema, aumentar la escalabilidad y reducir los costos operativos, además de permitir trasladar lógica de negocio sensible actualmente implementada en el frontend hacia entornos seguros como AWS Lambda.

\hfill

\textbf{Palabras clave:}  Ingeniería de software, Microservicios, Enterprise migration, Cloud computing, APIs, Vendor lock-in, Legaltech.
