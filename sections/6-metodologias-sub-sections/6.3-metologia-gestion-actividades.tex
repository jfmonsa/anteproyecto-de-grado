\subsection{Metodología de Gestión de Actividades}

La gestión de actividades se articula directamente con SCRUM, el cual fue seleccionado como el marco de trabajo que guiará la metodología de desarrollo de software. Bajo este enfoque el proyecto se organiza en iteraciones semanales fijas.

Para la planificación y control del trabajo se hace uso de un software interno de gestión de actividades, en el cual se encuentra el \textit{backlog} del producto, que representa el conjunto de tareas pendientes. A partir de este \textit{backlog}, se seleccionan las tareas y funcionalidades a priorizar en cada \textit{sprint} mediante una reunión sincrónica con el equipo (Desarrolladores, \textit{Product Owner} y  \textit{Tech Lead}). Adicionalmente, se utiliza un tablero Kanban como apoyo visual para facilitar el seguimiento del estado de cada tarea en tiempo real.

Por otro lado, la comunicación sincrónica se gestiona mediante Google Meet para realizar el sprint planning y Discord usado por el equipo de desarrollo para hacer Pair Programming y Revisión de código. La comunicación asincrónica se gestiona principalmente mediante Telegram y WhatsApp, herramientas que permiten mantener una interacción constante entre los miembros del equipo.
